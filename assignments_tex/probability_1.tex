\documentclass[a4paper]{jsarticle} %A4: 21.0 x 29.7cm
\usepackage{amsmath,amsfonts,amssymb,amsthm}
\usepackage{mathtools}
\usepackage{xcolor}
\usepackage{mathtools}
\usepackage{otf}
\usepackage{xspace}
\usepackage{newpxtext}
\renewcommand{\abstractname}{注意事項}
% \renewcommand{\qed}{\unskip\nobreak\quad\qedsymbol}
\newtagform{textbf}[\textbf]{[}{]}
\usetagform{textbf}
\newcommand*{\ie}{\textbf{\textit{i.e.}}\@\xspace}



\title{\vspace{-4cm}確率論のレポート問題}
\author{YI Ran-21122200512}
\date{\today}

\begin{document}
\maketitle


\begin{abstract}
  \textcolor{red}{A4で1枚にまとめること、提出は12月19日か1月16日のいずれかの授業中}  
\end{abstract}
\section*{問題}
\subsection*{問1}
$X,Y$がそれぞれポアソン分布 $P_o(\lambda),P_{o}(\mu)$ に従い、独立であるとする。ただし、
 $\lambda ,\mu >0$ とする。なお、確率変数 $X$ がポアソン分布$P_{o}(\lambda)$ に従うとは、確率関数 $f(k)=P(X=k)$ が次で与えられる。
\begin{equation*}
  f(k)=P(X=k)=e^{-\lambda}\dfrac{\lambda^k}{k!}\qquad (k=0,1,2,\dots)
\end{equation*}
この時、以下を示せ
\begin{enumerate}
  \item 和$ S=X+Yはポアソン分布 P_0(\lambda+\mu) $に従う
  \item 平均 $\mathbb{E}[X]=\lambda$
  \item 分散$ \mathtt{var}[X]=\lambda$
\end{enumerate}
\vspace{-0.4cm}
\section*{解答}
\begin{proof}[\textbf{proof\quad 1}]
  $X,Y$がそれぞれポアソン分布 $P_o(\lambda),P_{o}(\mu)$ によって、
  $$
    P(X=k)=e^{-\lambda}\dfrac{\lambda^k}{k!},\quad P(Y=k)=e^{-\mu}\dfrac{\mu^k}{k!}
  $$
  $XとYは独立なので,S=X+Yとn=0,1,2,\dots$により
\begin{flalign*}
    P(S=n)&=\sum_{k=0}^{n}P(X=k)P(Y=n-k)&\\
          &=\sum_{k=0}^{n}(e^{-\lambda}\dfrac{\lambda^k}{k!})(e^{-\mu}\dfrac{\mu^{n-k}}{(n-k)})&\\
          &=e^{-(\lambda+\mu)}\sum_{k=0}^{n}\dfrac{\lambda^k\mu^{n-k}}{k!(n-k)!}&\\
          &=\dfrac{1}{n!}e^{-(\lambda+\mu)}\sum_{k=0}^{n}\dfrac{n!\lambda^k\mu^{n-k}}{k!(n-k)!}&\\
          &=\dfrac{1}{n!}e^{-(\lambda+\mu)}\sum_{k=0}^{n}\tbinom{n}{k}\lambda^k\mu^{n-k}&\\
\end{flalign*} 
\vspace{-1.4cm}\\

二項定理より、$\sum_{k=0}^{n}\binom{n}{k}\lambda^k\mu^{n-k}=(\lambda+\mu)^n$\\
\quad したがって
  $$
    P(S=n)=e^{-(\lambda+\mu)}\dfrac{(\lambda+\mu)^n}{n!}\qquad (n=0,1,2,\dots)
  $$
\end{proof}
\vspace*{-2.2cm}
\begin{proof}[\textbf{proof\quad 2}]
\quad\\
$ \mathbb{E}[X]=\sum_{k=0}^{\infty}kP(X=k) $により、
\begin{flalign*}
  \mathbb{E}[X]=\sum_{k=0}^{\infty}kP(X=k)&=\sum_{k=0}^{\infty}ke^{-\lambda}\dfrac{\lambda^k}{k!}&\\
               &=\sum_{k=1}^{\infty}ke^{-\lambda}\dfrac{\lambda^k}{k!}&\\
               &=\lambda\sum_{k=1}^{\infty}e^{-\lambda}\dfrac{\lambda^{k-1}}{(k-1)!}&\\
\end{flalign*}
ここで、$l=kー1$とすると
\begin{align*}
\mathbb{E}[X]=\lambda\sum_{k=1}^{\infty}e^{-\lambda}\dfrac{\lambda^{k-1}}{(k-1)!}=\lambda\sum_{l=0}^{\infty}e^{-\lambda}\dfrac{\lambda^l}{l!}=\lambda\sum_{l=0}^{\infty}P(X=l)=\lambda\tag{2.1}\label{eq:2.1}
\end{align*}
となる.\\
\ie\quad $ \mathbb{E}[X]=\lambda $
\end{proof}


\begin{proof}[\textbf{proof\quad 3}]

\begin{flalign*}
  \mathtt{var}[X]&=\mathbb{E}[(X-\mathbb{E}[X])^2]=\mathbb{E}[X^2-2X \mathbb{E}[X]+\mathbb{E}[X]^2]&\\
                 &=\mathbb{E}[X^2]-2\mathbb{E}[X]^2+\mathbb{E}[X]^2=\mathbb{E}[X^2]-\mathbb{E}[X]^2&\\
                 &=\mathbb{E}[X(X-1)]+\mathbb{E}[X]-\mathbb{E}[X]^2\tag{3.1}\label{eq3.1}&
\end{flalign*}\\
$\because \mathbb{E}[X(X-1)]=\sum_{k=0}^{\infty}k(k-1)e^{-\lambda}\dfrac{\lambda^k}{k!}$\\
\vspace*{-0.1cm}
\begin{flalign}
  \therefore\mathbb{E}[X(X-1)]&=\sum_{k=2}^{\infty}k(k-1)e^{-\lambda}\dfrac{\lambda^2\lambda^{k-2}}{k(k-1)(k-2)!}\notag &\\
                    &=\lambda^2\sum_{k=2}^{\infty}e^{-\lambda}\dfrac{\lambda^{k-2}}{(k-2)!}\notag &
\end{flalign}
ここで、$j=k-2$とすると
\vspace*{-0.1cm}
\begin{flalign*}
  \mathbb{E}[X(X-1)]&=\lambda^2\sum_{j=0}^{\infty}e^{-\lambda}\dfrac{\lambda^j}{j!}=\lambda^2e^{-\lambda}\sum_{j=0}^{\infty}\dfrac{\lambda^j}{j!} となる &
\end{flalign*}
$\because\sum_{j=0}^{\infty}\dfrac{\lambda^j}{j!}=e^\lambda$\\
\quad\\
$\therefore \mathbb{E}[X(X-1)]=\lambda^2e^{-\lambda}e^{\lambda}=\lambda^2$\\
\quad\\
次に$\eqref{eq:2.1}式の\mathbb{E}[X]=\lambda と\mathbb{E}[X(X-1)]=\lambda^2$を\eqref{eq3.1}式に代入すると\\
$$
\mathtt{var}[X]=\mathbb{E}[X(X-1)]+\mathbb{E}[X]-\mathbb{E}[X]^2=\lambda^2+\lambda-\lambda^2=\lambda
$$
となる.\\
\ie\quad $\mathtt{var}[X]=\lambda$ 

\end{proof}



\end{document}



