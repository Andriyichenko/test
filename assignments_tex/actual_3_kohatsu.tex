\documentclass[a4paper]{jsarticle}
\usepackage{amsmath,amsfonts,amssymb,amsthm}
\usepackage{mathtools}
\usepackage{xcolor}
\usepackage{otf}
\usepackage{xspace}
\usepackage{newpxtext}
\usepackage{empheq,cases}
% \renewcommand{\abstractname}{注意事項}
% \renewcommand{\proofname}{\textbf{証明}}
% \renewcommand{\qed}{\unskip\nobreak\quad\qedsymbol}
\newtagform{textbf}[\textbf]{[}{]}
\usetagform{textbf}
\newcommand*{\ie}{\textbf{\textit{i.e.}}\@\xspace}
% \usepackage[a4paper, top=2cm, bottom=2cm, left=2cm, right=2cm]{geometry}
\title{\vspace{-4cm}アクチュアリーのレポート問題}
\author{YI Ran-21122200512}
\date{\today}
\begin{document}
\maketitle
%%%概要を出力したい人はこのように記述
\vspace{-0.4cm}
\section*{解答}
\begin{proof}[\textbf{proof\quad 1.1}]
  二つの独立なポアソン過程 $N^1$ と $N^2$ があり、それぞれのレートが $\lambda^1$ と $\lambda^2$ です。これらのジャンプ時刻は   
  \[  
  T_n^i = \sum_{j=1}^n \tau_j^i  
  \]  
で与えられ、${\tau_j^1 , j \in \mathbb{N}} と {\tau_j^2 , j \in \mathbb{N}}$ は独立である。\\
$X = N_t^1, Y = N_t^2$ とし、それぞれ独立な $\mathrm{Poisson}(\lambda^1 t)$, $\mathrm{Poisson}(\lambda^2 t)$ に従う。このときの合計 $X+Y$ について、確率母関数$G$が  

  \[  
  G_{X+Y}(z) = E[z^{X+Y}] = E[z^X z^Y] = E[z^X]E[z^Y] \quad (|z| < 1)
  \]  
ここで、頻度パラメータ$\lambda t$のポアソン分布の母関数は、\\  
$$ P(X = k) = \dfrac{(\lambda t)^n}{n!}e^{-\lambda t} $$
よって、\\

%左右两边写公式可以用$$\begin{aligned} 内容\quad 内容 \end{aligned}$$
\[  
\begin{aligned}  
\mathbb{E}[z^X] &= \sum_{k=0}^{\infty} z^k \frac{(\lambda^1 t)^k}{k!} e^{-\lambda^1 t} \\
                &= e^{-\lambda^1 t}\sum_{k=0}^{\infty}\frac{(\lambda^1 t z)^k}{k!} \\
                &= e^{-\lambda^1 t} e^{\lambda^1 t z}\\
                &= e^{\lambda^1 t(z-1)}  
\end{aligned}  
\hspace{6em}  %\quad or \qquad,1em:1个空格的意思
\begin{aligned}  
\mathbb{E}[z^Y] &= \sum_{k=0}^{\infty} z^k \frac{(\lambda^2 t)^k}{k!} e^{-\lambda^2 t} \\
                &= e^{-\lambda^2 t}\sum_{k=0}^{\infty}\frac{(\lambda^2 t z)^k}{k!} \\
                &= e^{-\lambda^2 t} e^{\lambda^2 t z}\\
                &= e^{\lambda^2 t(z-1)}  
\end{aligned}  
\]    
となる。\\
したがって、
\[
  G_{X+Y}(z) = e^{\lambda^1 t(z-1)} \times e^{\lambda^2 t(z-1)} = e^{(\lambda ^1 + \lambda ^2)t(z-1)}.  
\]  
これは $\mathrm{Poisson}((\lambda ^1 + \lambda ^2)t)$-分布の確率母関数と同じ形である。したがって、$X+Y$ の分布は $\mathrm{Poisson}((\lambda ^1 + \lambda ^2)t)$ である。
\end{proof}  

\begin{proof}[\textbf{proof\quad 1.2}]
  倒産理論では、企業や保険会社の収入と支出(損失)の確率過程をもとにモデル化される。
  損失の発生をポアソン過程とした場合、ある複数のリスク(複数のクレーム・災害・損害など)がポアソン過程で解析するなら、
  合計の損失到着はポアソンレートの単純な足し合わせとみなせる。
\end{proof}  

\newpage
\begin{proof}[\textbf{proof\quad 2}]
Cramer - Lundbergモデルの設定で、$U \sim e^{\mu^{-1}}$であるとき、

\[  
M_U(r) = E[e^{rU}] = \int_0^{\infty} e^{ru} \frac{1}{\mu} e^{-u/\mu} du = \frac{1}{\mu} \int_0^{\infty} e^{(r - \frac{1}{\mu})u} du\quad (r\in \mathbb{R}) 
\]  

ここで $\mu > 0$ かつ $r$ が十分小さければ積分は収束し、具体的に計算すると  

\[  
M_U(r) = \frac{1}{\mu} \left[ \frac{1}{r - \frac{1}{\mu}} e^{(r - \frac{1}{\mu})u} \right]_0^{\infty} = \frac{1}{\mu} \times \frac{1}{\frac{1}{\mu} - r} \quad (\text{ただし } r < \frac{1}{\mu}).  
\]  

したがって、指数分布の場合の調整係数を求める方程式は  

\[  
cr = \lambda \left( \frac{1}{1 - \mu r} - 1 \right),  
\]  

となる。これを解くと  

\[  
cr = \lambda \left( \frac{1 - (1 - \mu r)}{1 - \mu r} \right) = \lambda \left( \frac{\mu r}{1 - \mu r} \right).  
\]  

よって、
\begin{flalign*}
cr(1 - \mu r) &= \lambda \mu r\\   
cr - c\mu r^2   &= \lambda \mu r\\  
-c\mu r^2 + cr - \lambda \mu r &= 0\\  
-c\mu r^2 + (c - \lambda \mu)r &= 0  
\end{flalign*}
この2次方程式の$r^2$項の係数が負なので、正値$r>0$を求めると
$$ 
  r \{-c\mu r + (c-\lambda\mu)\} = 0
$$
したがって、
% \[  
% \begin{numcases}  
% r = 0 \\
% -c \mu r + (c - \lambda) \mu = 0
% \end{numcases}  
% \] 
\begin{numcases}
  {}
  r = 0 \\
  -c \mu r + (c - \lambda) \mu = 0\label{quad}
\end{numcases}
式[\ref{quad}] により、
$$
-c\mu r = \lambda\mu - c,r = \dfrac{c-\lambda\mu}{c\mu}
$$

$c>\lambda\mu$を仮定すると、\ie\quad $ \dfrac{c-\lambda\mu}{c\mu} $は正になる。\\
したがって、
$$ r = \dfrac{c-\lambda\mu}{c\mu}, \quad\ell (r) = cr - \lambda (M_U(r)-1) = 0$$
$\ell(r) = 0$をみたす 正の解$r = \gamma$ を調整係数とすると\\
$$ -\gamma = \dfrac{\lambda\mu - c}{c\mu}$$
となる。


\end{proof}
\end{document}